\section{VPN}

\begin{frame}{VPN}
Wie die Wireless Mesh Verbindungen zustande kommen geht aus dem
Aufbau der Firmware bereits hervor. Wie jedoch die Knoten
untereinander verbunden werden, soll der Vortrag in einem weiteren
Abschnitt über das verwendete VPN zeigen. Dazu gehört auch die
Aufteilung in Subnetze und die Verbindung der Gateways
untereinander.
\end{frame}

\begin{frame}{Allgemeines}
    \begin{itemize}
        \item Verwendetes VPN: fastd
        \item Wir nutzen keine Verschlüsselung (! :-O)
    \end{itemize}
    \begin{block}{fastd}
        % todo: was ist fastd ?
        There are no server and client roles defined by the
        protocol, this is just defined by the usage.
        \begin{itemize}
            \item Only one instance of the daemon is needed on each
                host to create a full mesh
            \item If no full mesh is established, a routing protocol
                is necessary to enable hosts that are not connected
                directly to reach each other
        \end{itemize}
    \end{block}
\end{frame}

\begin{frame}{Allgemeines}
    \begin{itemize}
        \item fastd Integration durch
        \begin{itemize}
            \item fastdstart.sh auf der Client-Seite
                \footnote{bsp/default/root\_file\_system/etc/fastdstart.sh.tpl}
            \item \$project\_\$hood\_fastd.sh auf der Server-Seite
                \footnote{auf den Gateways (!) Todo: befreien!}
            \item VPN-KeyXchange als Schlüsseltausch
                \footnote{\url{
                    http://git.freifunk-ol.de/projects/ffol/main.git}Todo refresh!}
        \end{itemize}
        \item Aufteilung in ,,hood''s:
        \begin{itemize}
            \item Stellt ein Layer-II Netz dar
            \item Ein Gateway kann mehrere Layer-II Netze bedienen
        \end{itemize}
    \end{itemize}
\end{frame}

\begin{frame}{\alt<1>{fastdstart.sh}{\$project\_\$hood\_fastd.sh}}
    \begin{itemize}
        \item \only<1>{Testet Internet-Connectivität}
        \item Legt fastd Konfiguration
            /etc/fastd/\$project\only<2>{.\$hood}/ \only<1>{im tmpfs} an:
            \footnote{\$project ist bei uns immer ,,fff''.}
            \begin{itemize}
                \item \$project.conf
                \item up.sh
                \item down.sh
                \item peers/
            \end{itemize}
        \item Erzeugt Pub/Priv-Keypaar
        \item Startet fastd
        \item Meldet sich beim VPN-KeyXchange an
        \item Lädt Liste mit Peers
        \item Refresht fastd
        \item \only<2>{Löscht verwaiste Peers}
    \end{itemize}
\end{frame}

\begin{frame}{VPN-KeyXchange}
    \only<1>{
        \begin{block}{HTTP Schnittstelle}
            \texttt{\tiny
                http://mastersword.de/\textasciitilde{}reddog/\$project/?mac=\$mac\&name=\$hostname\&port=\$port\&key=\$pubkey
            }
        \end{block}

        \begin{block}{Rückmeldung}
            \texttt{\tiny
                \#\#\#\#romauplink.conf\\
                \#name "romauplink";\\
                key "9a8ee8b797eed5d3b06778c47fe670987a0eda791ca557da56e6198be45f24c6";\\
                remote ipv4 "109.163.229.254" port 10000 float;\\[2ex]
                \#\#\#\#fff.conf\\
                \#name "fff";\\
                key "543ebc33b36210b10edf62fdb560e3ceac6c64b4ed9ad852f39954522934cd8a";\\
                remote ipv4 "192.168.30.23" port 10000 float;\\[2ex]
                \#\#\#\#90F652F45C34.conf\\
                \#name "ChamHH8Dachboden1";\\
                key "d0e4900fae535d189ef070a527dd00ce2688f3ff7f2b31c4f2846cb658b855fb";\\
                remote ipv4 "93.194.173.193" port 10000 float;\\
            }
        \end{block}
    }
\end{frame}
