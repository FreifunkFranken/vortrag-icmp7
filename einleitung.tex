\section{Einleitung}

\begin{frame}{Einleitung}
	\begin{itemize}
		\item<1-> Freifunk Franken ist lokaler Ableger der Freifunk-Bewegung (freifunk.net)
		\item<1-> nicht-kommerzielle Initiative für freie Funknetzwerke\\
							\begin{itemize}
								\item[$\rightarrow$] Bürger investieren in Eigenregie Zeit, Geld und Enthusiasmus
							\end{itemize}
 		\item<1-> nicht nur \glqq kostenloses Internet\grqq $\Rightarrow$ \glqq freies Netzwerken\grqq\\
							\begin{itemize}
								\item<2-> lokal intressante Dienste zur Verfügung stellen (Webcams)
								\item<2-> Text, Musik und Filme über das interne Freifunk-Netz übertragen
								\item<2-> über lokale Dienste Chatten oder Telefonieren
							\end{itemize}
	\end{itemize}
\end{frame}

\begin{frame}{Einleitung}
	wie es funktioniert:
	\begin{itemize}
		\item<1-> Freifunker stellen einen WLAN-Router für sich selbst und den Datentransfer der anderen Teilnehmer zur Verfügung
							\begin{itemize}
								\item<1-> ggf. mit Anschluss an das www
							\end{itemize}
		\item<1-> benachbarte Router verbinden sich und spannen ein sogenanntes Mesh-Netzwerk auf
		\item<1-> nicht benachbarte Router verbinden sich mittels VPN-Tunnel zum Freifunk Franken-Server
		\item<1-> jegliche Verbindung ins www wird hierrüber umgeleitet, um Risiken der Störerhaftung zu entgehen
	\end{itemize}
\end{frame}

\begin{frame}{Einleitung}
	für den einzelnen Freifunker ist hierzu nötig:
	\begin{itemize}
		\item<1-> ein günstiger, unterstützter Router (ab ca. 17€)
		\item<1-> eine spezielle Linuxdistribution, die Firmware
		\item<1-> die Zustimmung zum \glqq Pico-Peering Agreement\grqq
							\begin{itemize}
								\item<2-> Regelwerk, das grundsätzliche Eigenschaften eines freien Netzwerkes sichert
													\begin{enumerate}
														\item Freier Transit
														\item Offene Kommunikation
														\item Keine Garantie (Haftungsausschluss)
														\item Nutzungsbestimmungen
														\item Lokale (individuelle) Zusätze
													\end{enumerate}
							\item<2-> die Freifunk Firmware implementiert diese Grundsätze standardmäßig
							\end{itemize}
	\end{itemize}
\end{frame}
