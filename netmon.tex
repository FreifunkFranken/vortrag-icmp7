\section{Netmon}

%Ein spezieller Dienst im Freifunk Franken Netz ist das
%Network-Monitoring, kurz Netmon. Netmon selbst soll nicht Teil des
%Vortrags sein, wohl aber die Verbindung zwischen Netmon und den
%Knoten. Dazu zählt zum Beispiel das Handling der Hostnames und das
%Einsammeln der Statusdaten.

\begin{frame}{Netmon}
    \begin{itemize}
        \item Nodewatcher
        \begin{itemize}
            \item Generiert Status-Daten
        \end{itemize}
        \item Configurator
        \begin{itemize}
            \item Verknüpft Netmon und Knoten
        \end{itemize}
        \item Crawler
        \begin{itemize}
            \item Sammelt Status-Daten
        \end{itemize}
        \item Netmon
        \begin{itemize}
            \item Visualisiert Status-Daten
        \end{itemize}
    \end{itemize}
\end{frame}

\begin{frame}{Nodewatcher}
    \begin{itemize}
        \item Erzeugt XML Datei
        \item Läuft alle 5 Minuten auf den Knoten
        \item node.data über Webinterface downloadbar
    \end{itemize}
\end{frame}

\begin{frame}{Configurator}
    \begin{itemize}
        \item Knoten kennt Netmon's Link-Local Adresse
        \item Knoten meldet alle 5 Minuten seine MAC Adresse ans Netmon
        \item Netmon meldet dabei zurück, dass der Knoten noch nicht eingetragen wurde
        \item Benutzer ,,übernimmt'' Knoten im Netmon
        \item (Benutzer gibt dem Knoten einen Namen)
        \item Knoten meldet wieder seine MAC an Netmon
        \item Netmon meldet router\_id, update\_hash und api\_key zurück
        \item Knoten trägt seine Link-Local Adresse im Netmon ein
        \item Netmon pollt einmal alle 10 Minuten nach Router-Daten
        \item Knoten pollt alle 5 Minuten nach seinem Hostname
    \end{itemize}
\end{frame}

\begin{frame}{Crawler}
    \begin{block}{Pollt alle Router jeweils alle 10 Minuten}
        \begin{itemize}
            \item Jeweils 20 Knoten nacheinander
            \item Ping (mit PsExecute für Timeout)
            \begin{itemize}
                \item Bohrt neigh Tabelle auf
                \item Eliminiert initial Timeout
            \end{itemize}
            \item curl'ed node.data
            \item Dekodiert xml
            \item Füllt MySQL Tabellen
        \end{itemize}
    \end{block}
\end{frame}

\begin{frame}{Netmon}
    \begin{itemize}
        \item Wie sieht Netmon aus?
        \begin{itemize}
            \item[$\rightarrow$] \url{http://netmon.freifunk-franken.de}
        \end{itemize}
    \end{itemize}
\end{frame}
