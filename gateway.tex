\section{Gateway}

%Über die Gateways steht i.d.R. auch eine Internetverbindung zur
%Verfügung. Auch die Installation dieser Gateways sollen daher Teil
%des Vortrags sein.

\begin{frame}{Gateway}
    \begin{block}{Aktueller Zustand}
        \begin{itemize}
            \item Gateway in Rumänien
            \begin{itemize}
                \item[$\rightarrow$] Direkt Internet
            \end{itemize}
            \item Gateway in Nürnberg
            \begin{itemize}
                \item[$\rightarrow$] OpenVPN über Berlin
            \end{itemize}
        \end{itemize}
    \end{block}

    $\longrightarrow$ Das Beispiel hier stammt vom Gateway in Nürnberg
\end{frame}

\begin{frame}{Gateway}
    \begin{block}{Voraussetzungen}
        \begin{itemize}
            \item Server (kein Spielzeug; also kein Pi)
            \item Richtiges Betriebssystem (z.B. ein Debian)
            \item Policy Based Routing
            \item DHCP- DNS-Server
            \item batman + fastd (fff\_fuerth\_fastd.sh)
            \item Lokale Netz Konfiguration
            \item Freifunk-VPN (z.B. ip-ip)
            \item olsr
            \item Ein Tunnel um Traffic ab zu werfen\\
        \end{itemize}
    \end{block}
\end{frame}

\begin{frame}{Gateway}
    \begin{block}{Setup}
        \begin{itemize}
            \item IP Bereich
            \item .. was soll er tun? .. kurz Wiki durch gehen
        \end{itemize}
    \end{block}
\end{frame}

\begin{frame}[fragile]{Gateway}
    \begin{itemize}
        \item Policy Based Routing
        \item Trennung der Routing Tabellen
    \end{itemize}

    \begin{block}{/etc/iproute2/rt\_tables:}
        \scriptsize
        \begin{lstlisting}[gobble=12]
            10 fff
        \end{lstlisting}
    \end{block}
\end{frame}

\begin{frame}[fragile]{Gateway}
    \begin{block}{/etc/default/isc-dhcp-server:}
        \scriptsize
        \begin{lstlisting}[gobble=12]
            INTERFACES="bat1"
        \end{lstlisting}
    \end{block}

    \begin{block}{/etc/dhcp/dhcpd.conf:}
        \scriptsize
        \begin{lstlisting}[gobble=12]
            option domain-name "freifunk-franken.de";
            option domain-name-servers 10.50.16.1;
            authoritative;
            # fuerth
            subnet 10.50.32.0 netmask 255.255.248.0 {
                range 10.50.35.0 10.50.36.255;
                option routers 10.50.32.2;
                option domain-name-servers 10.50.32.2;
            }
        \end{lstlisting}
    \end{block}
\end{frame}

\begin{frame}[fragile]{Gateway}
    \begin{itemize}
        \item Bind
        \item Namesauflösung für Internet-Anfragen
    \end{itemize}

    \begin{block}{/etc/bind/named.conf.options:}
        \scriptsize
        \begin{lstlisting}[gobble=12]
            //listen-on-v6 { any; };
            allow-recursion { 10.50.0.0/16; };
        \end{lstlisting}
    \end{block}
\end{frame}


\begin{frame}[fragile]{Gateway}
    \begin{itemize}
        \item batman-adv-2013.4.0
        \item batctl-2013.4.0
        \item[$\rightarrow$] ./configure \&\& make install
    \end{itemize}

    \begin{block}{/etc/modules:}
        \scriptsize
        \begin{lstlisting}[gobble=12]
            batman-adv
        \end{lstlisting}
    \end{block}
\end{frame}

\begin{frame}[fragile]{Gateway}
    \begin{block}{/etc/default/fastd:}
        \scriptsize
        \begin{lstlisting}[gobble=12]
            AUTOSTART="none"
        \end{lstlisting}
    \end{block}
    \begin{block}{/etc/fff\_fuerth\_fastd.sh:}
        \scriptsize
        \begin{lstlisting}[gobble=12]
            %todo!
        \end{lstlisting}
    \end{block}
\end{frame}

\begin{frame}[fragile]{Gateway}
    \begin{block}{/etc/network/interfaces:}
        \scriptsize
        \begin{lstlisting}[gobble=12]
            auto ffffuerthVPN
            iface ffffuerthVPN inet manual
                post-up /usr/local/sbin/batctl -m bat1 if add $IFACE
                post-up ifconfig $IFACE up
                post-up ifup bat1
                post-down ifdown bat1
                post-down ifconfig $IFACE down
        \end{lstlisting}
    \end{block}
\end{frame}

\begin{frame}[fragile]{Gateway}
    \begin{block}{/etc/network/interfaces:}
        \scriptsize
        \begin{lstlisting}[gobble=12]
            auto bat1
            iface bat1 inet manual
                post-up ifconfig $IFACE up
                post-up ip addr add 10.50.32.2/21 dev $IFACE
                post-up ip rule add iif $IFACE table fff
                post-up ip rule add from 10.50.32.0/21 table fff
                post-up ip rule add to 10.50.32.0/21 table fff
                post-up ip route add 10.50.32.0/21 dev $IFACE table fff
                post-up invoke-rc.d isc-dhcp-server restart
                post-down ip route del 10.50.32.0/21 dev $IFACE table fff
                post-down ip rule del from 10.50.32.0/21 table fff
                post-down ip rule del to 10.50.32.0/21 table fff
                post-down ip rule del iif $IFACE table fff
                post-down ifconfig $IFACE down
        \end{lstlisting}
    \end{block}
\end{frame}

\begin{frame}[fragile]{Gateway}
    \begin{block}{/etc/network/interfaces:}
        \scriptsize
        \begin{lstlisting}[gobble=12]
            auto ro1
            iface ro1 inet static
                address 192.168.230.2
                pre-up iptunnel add ro1 mode gre \
                        local 193.192.41.129 \
                        remote 109.163.229.254 ttl 255
                up ifconfig ro1 multicast
                pointopoint 192.168.230.1
                post-down iptunnel del ro1
        \end{lstlisting}
    \end{block}
\end{frame}

\begin{frame}[fragile]{Gateway}
    \begin{block}{/etc/olsrd/olsrd.conf:}
        \scriptsize
        \begin{lstlisting}[gobble=12]
            Interface "ro1" {
                Ip4Broadcast 255.255.255.255
                LinkQualityMult 192.168.230.2 0.5
            }

            Hna4 {
                10.50.32.0 255.255.248.0
            }

            LoadPlugin "olsrd_httpinfo.so.0.1" {
                   PlParam "Port" "8080"
                   PlParam "Net" "0.0.0.0 0.0.0.0"
            }

            RtProto 8
            RtTable 10
            RtTableDefault 10
            RtTableTunnel 10
        \end{lstlisting}
    \end{block}
\end{frame}

