\section{Firmware}

%Nachdem der Aufbau der Firmware geklärt ist, soll ein kurzer Abriss
%über den Bauprozess der aktuellen Firmware gezeigt werden. Auch die
%letzten Entwicklungen der Franken eigenen Firmware werden gezeigt
%und erklärt. An dieser Stelle könnten Vorkenntnisse über OpenWrt
%Vorteilhaft sein, sind aber keine Voraussetzung für den Vortrag.

\begin{frame}[fragile]{Firmware}
    \only<1-3>{
        \begin{block}{Wie wird die Firmware gebaut?}
            \begin{itemize}
                \item<1-> git clone \sout{git://git.freifunk-ol.de/ffol/user/reddog/firmware.git}
                \item<1-> cd firmware
                \item<1-> ./buildscript
                \item<3-> ./buildscript selectbsp bsp/board\_wr1043nd.bsp
                \item<3-> ./buildscript selectcommunity community/franken.cfg
                \item<3-> ./buildscript
            \end{itemize}
        \end{block}
    }

    \only<2>{
        \texttt{\scriptsize
            Please select a Board-Support-Package using:\\
            ./buildscript selectbsp
        }
    }

    \only<4>{
        \texttt{\scriptsize
            Working with bsp/board\_wr1043nd.bsp and community/franken.cfg\\[2ex]
            This is the Build Environment Script of the Freifunk Community\\
            Oldenburg.\\
            Usage: ./buildscript command\\
            command:\\
            \hspace{1cm}selectcommunity [communityfile]\\
            \hspace{1cm}selectbsp [bsp file]\\
            \hspace{1cm}prepare\\
            \hspace{1cm}config\\
            \hspace{1cm}build\\
            \hspace{1cm}flash\\
            \hspace{1cm}download\\[2ex]
            If you need help to one of these options just type\\
            ./buildscript command help\\
        }
    }
\end{frame}

\begin{frame}[fragile]{Community File}
    \begin{block}{community/franken.cfg:}
        \scriptsize
        \begin{lstlisting}[gobble=12]
            BATMAN_CHANNEL=1
            ESSID_AP=franken.freifunk.net
            ESSID_MESH=batman.franken.freifunk.net
            BSSID_MESH=02:CA:FF:EE:BA:BE
            NETMON_IP=fe80::ff:feee:1
            VPN_PROJECT=fff
            NTPD_IP=fe80::ff:feee:1%br-mesh
            UPGRADE_PATH=http://[fe80::ff:feee:1%br-mesh]/dev/firmware/current/
        \end{lstlisting}
    \end{block}
\end{frame}

\begin{frame}[fragile]{Board-Support-Package}
    \begin{block}{bsp/board\_wr740n.bsp:}
        \scriptsize
        \begin{lstlisting}[gobble=12]
            machine=wr740n
            target=$builddir/$machine

            board_prepare() {
            }
            board_prebuild() {
            }
            board_postbuild() {
                cp $target/bin/ar71xx/openwrt-ar71[..]-squashfs-*.bin ./bin/
            }
            board_flash() {
                echo "nothing implemented"
            }
            board_clean() {
                /bin/rm -rf $target bin/*$machine*
            }
        \end{lstlisting}
    \end{block}
\end{frame}

\begin{frame}{Board-Support-Package}
    \begin{block}{bsp/wr740n/}
        \begin{itemize}
            \item .config
            \item root\_file\_system/etc/
            \begin{itemize}
                \item rc.local.board
                \item config/
                \begin{itemize}
                    \item board
                    \item network
                    \item system
                \end{itemize}
                \item crontabs/
                \begin{itemize}
                    \item root
                \end{itemize}
            \end{itemize}
        \end{itemize}
    \end{block}
\end{frame}

\begin{frame}{Was macht das Buildscript?}
    \begin{itemize}
        \item Das BSP file wird als dot Script eingeladen
        \item Generiert dynamisches sed-Script
        \begin{itemize}
            \item[$\rightarrow$] Templates mit richtigen Werten zu füllen
        \end{itemize}
    \end{itemize}
\end{frame}

\begin{frame}{Was passiert beim prepare?}
    \begin{itemize}
        \item Sourcen Downloaden
        \begin{itemize}
            \item[$\rightarrow$] separater Source Folder
            \item OpenWRT
            \item Sämtliche Packages
            \begin{itemize}
                \item[$\rightarrow$] ggfs werden Patches angewandt
            \end{itemize}
        \end{itemize}
        \item Ein ggfs altes Target wird gelöscht
        \item OpenWRT wird ins Target exportiert (kopiert)
        \item Eine OpenWRT Feed-Config wird mit dem lokalen Source Verzeichnis als Quelle angelegt
        \item Die Feeds werden geladen
        \item Spezielle Auswahl an Paketen wird geladen
        \item Patches werden angewandt
        \item board\_prepare wird aufgerufen (wird. z.B. für Patches für eine bestimmte HW verwendet)
    \end{itemize}
\end{frame}

\begin{frame}{Was passiert beim build?}
    \begin{itemize}
        \item prebuild
        \begin{itemize}
            \item \$target/files aufräumen
            \item Files aus default-bsp und bsp kopieren
            \item OpenWRT- und Kernel-Config kopieren
            \item board\_prebuild
            \item Templates transformieren
            \item Versionen speichern: \$target/files/etc/firmware\_release
        \end{itemize}
        \item OpenWRT: make
        \item postbuild
        \begin{itemize}
            \item board\_postbuild
        \end{itemize}
    \end{itemize}
\end{frame}

\begin{frame}{Config Targets?}
    \begin{itemize}
        \item prebuild
        \item OpenWRT: make menuconfig ; make kernel\_menuconfig
        \item Speichern, y/n?
        \item Config-Format vereinfachen
        \item Config ins BSP zurück speichern
    \end{itemize}
\end{frame}

